% (c) 2002 Matthew Boedicker <mboedick@mboedick.org> (original author) http://mboedick.org
% (c) 2003-2007 David J. Grant <davidgrant-at-gmail.com> http://www.davidgrant.ca
% (c) 2008 Nathaniel Johnston <nathaniel@nathanieljohnston.com> http://www.nathanieljohnston.com
%
%This work is licensed under the Creative Commons Attribution-Noncommercial-Share Alike 2.5 License. To view a copy of this license, visit http://creativecommons.org/licenses/by-nc-sa/2.5/ or send a letter to Creative Commons, 543 Howard Street, 5th Floor, San Francisco, California, 94105, USA.

\documentclass[letterpaper,11pt]{article}
\newlength{\outerbordwidth}
\pagestyle{empty}
\raggedbottom
\raggedright
%\usepackage[francais]{babel}
\usepackage[utf8]{inputenc}
\usepackage[T1]{fontenc}
\usepackage[svgnames]{xcolor}
\usepackage{framed}
\usepackage{tocloft}
\usepackage{etoolbox}
\usepackage{hyperref}
\usepackage{fontawesome}

\robustify\cftdotfill


%-----------------------------------------------------------
%Edit these values as you see fit

\setlength{\outerbordwidth}{3pt}  % Width of border outside of title bars
\definecolor{shadecolor}{gray}{0.75}  % Outer background color of title bars (0 = black, 1 = white)
\definecolor{shadecolorB}{gray}{0.93}  % Inner background color of title bars


%-----------------------------------------------------------
%Margin setup

\setlength{\evensidemargin}{-0.25in}
\setlength{\headheight}{0in}
\setlength{\headsep}{0in}
\setlength{\oddsidemargin}{-0.25in}
\setlength{\paperheight}{11in}
\setlength{\paperwidth}{8.5in}
\setlength{\tabcolsep}{0in}
\setlength{\textheight}{9.5in}
\setlength{\textwidth}{7in}
\setlength{\topmargin}{-0.3in}
\setlength{\topskip}{0in}
\setlength{\voffset}{0.1in}


%-----------------------------------------------------------
%Custom commands
\newcommand{\resitem}[1]{\item #1 \vspace{-2pt}}
\newcommand{\resheading}[1]{\vspace{8pt}
  \parbox{\textwidth}{\setlength{\FrameSep}{\outerbordwidth}
    \begin{shaded}
\setlength{\fboxsep}{0pt}\framebox[\textwidth][l]{\setlength{\fboxsep}{4pt}\fcolorbox{shadecolorB}{shadecolorB}{\textbf{\sffamily{\mbox{~}\makebox[6.762in][l]{\large #1} \vphantom{p\^{E}}}}}}
    \end{shaded}
  }\vspace{-5pt}
}
\newcommand{\ressubheading}[4]{
\begin{tabular*}{6.5in}{l@{\cftdotfill{\cftsecdotsep}\extracolsep{\fill}}r}
    \textbf{#1} & #2 \\
    \textit{#3} & \textit{#4} \\
\end{tabular*}\vspace{-6pt}}
%-----------------------------------------------------------


\begin{document}

\begin{tabular*}{7in}{l@{\extracolsep{\fill}}r}
\textbf{\Large Antoine Lambert}\\
born on 27/02/1985\\
\textbf{\large Software engineer}\\
\faEnvelopeO : antoine.lambert33@gmail.com & xx xxx xxxxxxxx\\
\faMobile : +33 ? ?? ?? ?? ?? & xxxx xxxxx xxxxxx\\
\url{https://scholar.google.fr/citations?user=wRrEbaAAAAAJ&hl} & 75002 Paris\\
\url{https://github.com/anlambert} & France\\
\end{tabular*}
\\

%%%%%%%%%%%%%%%%%%%%%%%%%%%%%%
\resheading{Education}
%%%%%%%%%%%%%%%%%%%%%%%%%%%%%%
\begin{itemize}

\item

  \ressubheading{University of Sciences and Technologies Bordeaux 1}{Talence, France}{PhD in Computer Science, in the field of Information Visualization}{2009 - 2012}

  \begin{itemize}

    \resitem{Title of the thesis : \emph{Interactive graph visualization : elaboration and optimisation of algorithms with high computationnal cost.}}

    \resitem{Host Laboratory : LaBRI (Laboratoire Bordelais de Recherche en Informatique)}

    \resitem{Very honorable mention}

    \resitem{Online manuscript (in french) : \url{http://www.theses.fr/2012BOR14664}}

  \end{itemize}

\item

  \ressubheading{University of Sciences and Technologies Bordeaux 1}{Talence, France}{Professional Master of Computer Science, Software Engineering specialty}{2006 - 2008}

  \begin{itemize}

    \resitem{With honors}

  \end{itemize}

\item

  \ressubheading{University of Sciences and Technologies Bordeaux 1}{Talence, France}{Bachelor in Computer Science}{2003 - 2006}

  \begin{itemize}

    \resitem{With honors}

  \end{itemize}

\end{itemize}

%%%%%%%%%%%%%%%%%%%%%%%%%%%%%%
\resheading{Experience}
%%%%%%%%%%%%%%%%%%%%%%%%%%%%%%
\begin{itemize}

\item
  \ressubheading{Thales Services SAS}{Vélizy Villacoublay, France}{Software engineer}{Since September 2013}
  
  Currently working in a research and development laboratory of Thales Communications \& Security called CenTAI (Centre de Traitement et d’Analyse de l’Information), specialized in Big Data, Big Analytics and Visual Analytics. 
  \begin{itemize}
    \item{Responsible of visualization works in the Big Data projects of the laboratory.}
    \item{Developer and maintainer of a large graphs visualization system for a project named OSIntLab, a social networks survey framework developed by Thales.}
    \item{Development of a large graph visualization solution for the web by porting the \emph{Tulip} framework to Javascript.}
    \item{Development of numerous visualizations and map components (using D3.js and Leaflet) for the analytics web portal used internally in the laboratory (powered by Angular and based on Elasticsearch for the data source).}
    \item{Technological survey on contemporary web development to ease the building and the performance of complex client side applications}
  \end{itemize}

\item

  \ressubheading{Universiy of Sciences and Technologies Bordeaux 1, LaBRI}{Talence, France}{Research Engineer}{Mach 2013 - August 2013}
  \begin{itemize}
    \resitem{Technical assistance in software development for research projects}
    \resitem{Maintenance and evolution of the \emph{Tulip} visualization framework}
  \end{itemize}

\item
  \ressubheading{University of Sciences and Technologies Bordeaux 1, LaBRI}{Talence, France}{Temporary Lecturer and Research Assistant}{September 2012 - February 2013}
  \begin{itemize}
    \resitem{Teaching activities at the University Institute of Technology Bordeaux 1}
    \resitem{Subjects taught : }
    \begin{itemize}
      \resitem{Introduction to Unix-like Operating Systems}
      \resitem{Introduction to Algorithmic and Programming}
    \end{itemize}
  \end{itemize}

  \item
    \ressubheading{LaBRI}{Talence, France}{Software engineer}{October 2008 - October 2009}
    \begin{itemize}
      \resitem{Integrated in a development team to work on an open source graph visualization framework named \emph{Tulip}}
      \resitem{Development of new visualisation components for the framework using Qt and OpenGL}
    \end{itemize}

  \item
    \ressubheading{Atos Origin Integration}{Pessac, France}{Intern}{Avril 2008 - Septembre 2008}

    Internship in an application management team for the ERP project, on behalf of France Telecom, named "New Convergence" and based on 
    Oracle Applications.
    

  \item
    \ressubheading{I2S}{Pessac, France}{Intern}{Juillet 2007 - Août 2007}

    Elaboration and implementation of computer vision algorithms for the detection and the tracking of a soccer cage in a game video stream.

\end{itemize}

%%%%%%%%%%%%%%%%%%%%%%%%%%%%%%
\resheading{Compétences}
%%%%%%%%%%%%%%%%%%%%%%%%%%%%%%
\begin{itemize}
\item Grande expérience en développement informatique : du logiciel natif à l'application web
\begin{itemize}
  \item Langages de programmation : C/C++, Java, Python, JavaScript (\emph{fullstack} : côtés client et serveur), HTML, CSS, Scripts shell Unix, \ldots
  \item Outils de gestion de version : Subversion, Git, \ldots
  \item Outils de développement C/C++: CMake, Valgrind, GDB, \ldots
  \item Outils de développement JavaScript : Grunt, Gulp, Webpack, Npm, Emscripten, \ldots
  \item Bibliothèques/API C/C++ : Qt, OpenGL, Tulip, Python, \ldots
  \item Bibliothèques/API JavaScript : Angular, Lodash, Bootstrap, D3, WebGL, Leaflet, \ldots
  \item Technologies Big Data : Hadoop, HDFS, Spark, Elasticsearch, Kafka, \ldots
  \item Outils de documentation : Sphinx, Markdown, doxygen, \LaTeX, \ldots
  \item IDE : Qt Creator, Eclipse, Atom, \ldots
  \item Expert en développement C++ multi-plateformes (Linux, MacOS, Windows, Web)
\end{itemize}
\item Connaissances approfondies des techniques de visualisation de données : de la théorie à l'implémentation
\begin{itemize}
  \item Thèse réalisée sur le thème de la Visualisation d'Information
  \item Nombreuses contributions de recherche sur le sujet (plus d'une dizaine de publications dont un papier cité 82 fois)
  \item Développement et maintenance de plusieurs systèmes de visualisation, en particulier le framework \emph{open source} \emph{Tulip} dédié à l'analyse et la visualisation de grands graphes
\end{itemize}
\item Grand intérêt pour l'infographie et l'exploitation du processeur graphique (expert OpenGL/WebGL)
\item Grande connaissance du monde de l'\emph{open source}
\begin{itemize}
  \item utilisation quotidienne d'une distribution GNU/Linux (sécurité, stabilité, productivité)
  \item grande expérience dans la récupération, l'installation des dépendances et la construction de logiciels \emph{open source}
  \item contributeur dans différents projets \emph{open source}
\end{itemize}
\end{itemize}

%%%%%%%%%%%%%%%%%%%%%%%%%%%%%%
\resheading{Publications scientifiques (indice h: 8)}
%%%%%%%%%%%%%%%%%%%%%%%%%%%%%%

\ressubheading{Revues internationales avec comité de lecture et actes}{}{}{}

\begin{itemize}
 \item David AUBER, Charles HUET, Antoine LAMBERT, Benjamin RENOUST, Arnaud SALLABERY et Agnes SAULNIER. "Gospermap : Using a gosper curve for laying out hierarchical data". Dans : \underline{IEEE Transactions on Visualization and Computer Graphics} 19.11 (2013), p. 1820-1832
 \item Antoine LAMBERT, Jonathan DUBOIS et Romain BOURQUI. "Pathway Preserving Representation of Metabolic Networks". Dans : \underline{Computer Graphics Forum} 30.3 (2011), p. 1021-1030
 \item Antoine LAMBERT, Romain BOURQUI et David AUBER. "Winding Roads : Routing edges into bundles". Dans : \underline{Computer Graphics Forum} 29.3 (2010), p. 853-862
\end{itemize}

\ressubheading{Conférences internationales avec comité de lecture et actes}{}{}{}

\begin{itemize}
  \item Antoine LAMBERT et David AUBER. "Graphs analysis and visualization with Tulip-Python". Poster présenté à \underline{5th European Conference for Scientists using Python (EuroSciPy 2012)}. 2012
  \item Antoine LAMBERT, François QUEYROI et Romain BOURQUI. "Visualizing patterns in Node-link Diagrams". Dans: \underline{Proceedings of the 16th International Conference on Information Visualisation}. IV'12. IEEE Computer Society, 2012, p. 48-53
  \item Antoine LAMBERT, Romain BOURQUI et David AUBER. "3D Edge Bundling for Geographical Data Visualization". Dans : \underline{Information Visualisation (IV), 2010 14th International Conference}. IEEE Computer Society, 2010, p 329-335
  \item Antoine LAMBERT, David AUBER et Guy MELANÇON. "Living Flows : Enhanced Exploration of Edge-Bundled Graphs Based on GPU-Intensive Edge Rendering". Dans : \underline{Information Visualisation (IV), 2010 14th International Conference}. IEEE Computer Society, 2010, p 523-530
\end{itemize}

\ressubheading{Chapitre de livre}{}{}{}

  \begin{itemize}
    \item Antoine LAMBERT, Romain BOURQUI et David AUBER. "Graph visualization for geography". Dans : \underline{Methods for Multilevel Analysis and Visualisation of Geographical Networks}, p. 81-102. Springer Netherlands
  \end{itemize}

\ressubheading{Rapport de recherche}{}{}{}

  \begin{itemize}
    \item David AUBER, Daniel ARCHAMBAULT, Romain BOURQUI, Antoine LAMBERT, Morgain MATHIAUT, Patrick MARY, Maylis DELEST, Jonathan DUBOIS, Guy MELANÇON. "The Tulip 3 Framework : A Scalable Software Library for Information Visualization Applications". Rap. tech. RR-7860. INRIA, 2012, p. 31
  \end{itemize}

%%%%%%%%%%%%%%%%%%%%%%%%%%%%%%
\resheading{Contributions à des projets \emph{open source}}
%%%%%%%%%%%%%%%%%%%%%%%%%%%%%%

\begin{itemize}

\item \ressubheading{Tulip \url{http://www.tulip-software.org}}{C++, Python, OpenGL}{Framework dédié à l'analyse, au dessin et à la visualisation de grands graphes}{Depuis 2008}

\begin{itemize}
  \item Logiciel développé au sein du LaBRI à Bordeaux, très reconnu dans la communauté de la visualisation de graphes, disponible pour Windows, MacOS, Linux (paquets binaires disponibles sous Debian), FreeBSD
  \item Système stable et performant : plus de 15 ans de développement à son actif
  \item Contributions personnelles:
  \begin{itemize}
    \item intégration des algorithmes de dessin de graphes de la bibliothèque OGDF (\emph{Open Graph Drawing Framework}) \url{http://www.ogdf.net}
    \item développement du module Python pour Tulip et intégration d'un IDE Python plus un moteur d'exécution de script dans le logiciel \url{https://pypi.python.org/pypi/tulip-python}
    \item développement de nombreux plugins d'algorithmes, de visualisations et d'interactions : Delaunay triangulation, Voronoï diagram, Edge bundling, Polyomino Packing, Google Maps view, Histogram view, Parallel Coordinates view, Pixel oriented view, Scatter Plot view, Fisheye interactor, Graph Splatting Interactor, Lasso Selection interactor, Magnifying Glass interactor, Neighborhood Highlighter interactor
    \item portage en Javascript du framework via le compilateur \emph{Emscripten} pour la visualisation de grands graphes sur le Web \url{https://github.com/tulip5/tulip/tree/master/library/tulip-javascript} \url{https://anlambert.github.io/tulipjs/tulip_web.html}
  \end{itemize}
\end{itemize}

\item \ressubheading{Emscripten \url{https://github.com/kripken/emscripten}}{JavaScript, Python, C/C++}{Compilateur C/C++ vers JavaScript}{Depuis 2013}

\item \ressubheading{Auditwheel \url{https://github.com/pypa/auditwheel}}{Python}{Outil pour la distribution de modules Python binaires sous Linux}{Depuis 2016}

\end{itemize}

\end{document}
